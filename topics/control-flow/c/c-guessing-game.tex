\subsection{Implementing the Guess that Number in C} % (fold)
\label{sub:c_guessing_game}

\sref{sec:control_flow_using_these_concepts} of this Chapter introduced the `Guess that Number' program. This program contained a Function to \texttt{Perform Guess} and Procedures to \texttt{Print Line} and \texttt{Play Game}. Each of these involved some control flow in their logic, as shown in the Flowcharts in \sref{sec:control_flow_using_these_concepts}. The full C implementation of the Guess that Number program is shown in \lref{lst:storing-data-c-guessing-game}.

\straightcode{\ccode{lst:storing-data-c-guessing-game}{C code for the Guessing Game}{code/c/control-flow/guess-that-number.c}}

\mynote{
\begin{itemize}
  \item \texttt{stdlib.h} is needed for the \texttt{random} function and the \texttt{srandom} procedure.
  \item \texttt{time.h} is needed to get the current time used to seed the random number generator.
  \item \texttt{stdbool.h} gives access to the \texttt{bool} data type in C for \nameref{sub:boolean_data}.
  \item In this code \texttt{perform\_guess} is laid out differently to that shown previously. This is also an acceptable layout as it shows the different paths clearly. While it does not show the structure as well, it is a clean and neat way of presenting this code.
\end{itemize}
}

% subsection c_guessing_game (end)

