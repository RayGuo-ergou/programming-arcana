\clearpage
\subsection{Pascal If Statement} % (fold)
\label{sub:pas_if_statement}

The if statement is a \nameref{sub:branching} statement. This can be used to optionally run a block of code, providing two alternate paths controlled by a Boolean expression.

\passyntax{psynt:branching-if-statement}{an if statement}{control-flow/if-statement}

\passection{\pascode{plst-test-if}{Pascal if test code}{code/pascal/control-flow/TestIf.pas}}

\mynote{
\begin{itemize}
  \item This is the Pascal syntax for the \nameref{sub:if_statement}.
  \item The \texttt{then} keyword tells the compiler where the if's condition ends.
  \item Notice that the \texttt{else} branch is optional.
  \item When the expression is \texttt{True} the first path is taken.
  \item When the expression is \texttt{False} the else branch is taken.
  \item Notice that there is \textbf{no} semicolon (;) after the first statement before the else branch. 
\end{itemize}
}

% subsection c_if_statement (end)