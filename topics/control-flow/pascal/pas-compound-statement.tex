\clearpage
\subsection{Pascal Compound Statement} % (fold)
\label{sub:pas_compound_statement}

Most of the Pascal structured statements only allow a single statement within each path. For example, the paths in the two branches of an \nameref{sub:pas_if_statement} can only contain a single statement. The \nameref{sub:compound_statement} allows you to group together multiple statements within a single \emph{compound statement}.

\passyntax{psynt:branching-compound-statement}{a compound statement}{control-flow/compound-statement}

\passection{\pascode{plst-test-compound}{Pascal compound statement test code}{code/pascal/control-flow/TestCompound.pas}}

\mynote{
\begin{itemize}
  \item \fref{psynt:branching-compound-statement} shows the syntax for a \nameref{sub:compound_statement} in Pascal.
  \item The code in \lref{plst-test-compound} shows an if statement that includes two compound statements within its branches.
  \item The compound statement is a standard statement, and can be used anywhere a statement can appear. Its practical use is for grouping statements within other structured statements, and you are unlikely to find it used in any other way.
\end{itemize}
}

% subsection c_compound_statement (end)