\clearpage
\subsection{Pascal Case Statement} % (fold)
\label{sub:pas_case_statement}

The case statement allows you to switch between a number of paths.

\passyntax{psynt:branching-case-statement}{a case statement}{control-flow/case-statement}

\mynote{
\begin{itemize}
  \item This is the Pascal syntax to declare a \nameref{sub:case_statement}.
  \item The \emph{constant expressions} in each \emph{case} must be ordinal values (integers or characters).
  \item By using \texttt{constant..constant} the case will match any value in this range, e.g. \texttt{0..9}.
  \item The code in \lref{plst-test-case} shows an example use for a case statement.
  \item The \texttt{default} path is taken when none of the other paths match the expression.
  \item Each \emph{case} contain a single statement.
  \item Watch \url{http://www.youtube.com/watch?v=zIV4poUZAQo} for important details on the legendary Knights of Ni.
\end{itemize}
}

\passection{\pascode{plst-simple-case}{Pascal case test code with a character}{code/pascal/control-flow/SimpleCase.pas}}

\clearpage

\passection{\pascode{plst-test-case}{Pascal case test code with an integer}{code/pascal/control-flow/TestCase.pas}}

% subsection case_statement (end)