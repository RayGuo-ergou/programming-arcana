\clearpage
\subsection{The First Program: Hello World} % (fold)
\label{sub:hello world}

There is one very special program that all developers create. This is the first program a software developer creates when they start using a new language, or technology. It is the famous \textbf{Hello World}!

This is a very simple program, it runs and outputs the text `Hello World' to the Terminal. So, why is this the first program? It makes sure that everything is set up correctly. If `Hello World' does not work, then there is something wrong with your setup you need to check.

Listings \ref{lst:hello-world-c} and \ref{lst:hello-world-pas} show the code for the `Hello World' program written with the C and Pascal programming languages. Both programs result in the same output when run: they write the text `Hello World!' to the Terminal. They both use the same basic programming structures, and they both go about performing the task in the same way. At this stage, however, they are both just fancy text. What we need to do is use a special tool to convert these into \emph{programs}, we need to \textbf{compile} them.

\csection
{
\ccode{lst:hello-world-c}{Hello World code in C.}{code/c/program-creation/hello-world.c}
}

\passection{
 \pascode{lst:hello-world-pas}{Hello World code in Pascal.}{./topics/program-creation/pascal/HelloWorld.pas}
}

\mynote{
\begin{itemize}
  \item This source code needs to be written into a \textbf{text file}.
  \item C source code usually has a \textbf{.c} file extension. For example, \textbf{hello-world.c}.
  \item Pascal source code usually has a \textbf{.pas} file extension. For example \textbf{HelloWorld.pas}.
  \item It is a good idea to create a directory (folder) under which you will place your code. Within this directory you can create other directories for individual projects, or for the code related to the chapters in this book.
\end{itemize}
}

% section hello world (end)
