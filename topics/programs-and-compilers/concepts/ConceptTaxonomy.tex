\clearpage
\subsubsection{Programming Jargon and Concept Taxonomy} % (fold)
\label{sub:concept_taxonomy}

Programming has a lot of its own jargon. As you learn to develop software it is also important that you start to learn this \emph{special language} that software developers use to discuss their programs. You will find that this terminology is used in many places. It is used in programming texts, in discussions between developers, in discussion boards, blogs, anywhere that developers are discussing software development. Having a clear understanding of this terminology will help you make the most of these resources.

The concepts in this book are closely linked to this programming terminology. To help you understand each concept, we have classified them using one of the following categories:

\begin{itemize}
  \item \textbf{Artefact}: An artefact is something that you can create in your code.
  \item \textbf{Action}: Actions are things that you can \emph{command} the computer to do.
  \item \textbf{Term}: These are general terms, used to describe some aspect.
\end{itemize}

When you are reading about the different concepts in this book you can use these classifications to help you think about how you may use the knowledge you are gaining.

\subsubsection{Artefacts} % (fold)
\label{ssub:artefacts}

Artefacts are things that you create in your code. Programming is a very \emph{abstract} activity, you spend most of your time working with concepts and ideas. You write text, code, that will create things within the computer when your code is run. 

When you are learning about a new kind of artefact come up with ways of visualising it. It is a \textbf{thing} that you are creating with your code. Try to picture the artefact within your code. These artefacts are the basic building blocks that you have to work with. You need to be very familiar with them, how they work, and what you can do with them.

% subsubsection artefacts (end)

\subsubsection{Actions} % (fold)
\label{ssub:actions}

Actions get the computer to perform a task. Your actions will be coded within the \textbf{artefacts} that you create, and will define how artefacts behave when they are used. The actions themselves are commands that you issue to the computer. They are executed one at a time, and each kind of action gets the computer to carry out certain tasks.

When you are learning a new kind of action you need to see what this action does. To start with you should play with it, test it out, and see if you can understand what it is getting the computer to do. As you progress you need to start thinking about how you can sequence these actions so that the computer performs the tasks you want it to. There are only a very few kinds of actions, so it is by combining them that you can get the computer to do what you want. 

% subsubsection actions (end)

\subsubsection{Terms} % (fold)
\label{ssub:terms}

The remaining terms are words that developers use to explain concepts. These are not things that you create, or actions that you request. These are just words that you need to \emph{know}.

When you are learning a new term you need to try to commit it to memory. Memorise the terms, try to use them in sentences, explain them to others. All of these tasks will help you understand, and remember these terms.

% subsubsection terms (end)

\bigskip

Now that the book overview is done, lets get started with the real material\ldots

% subsection concept_taxonomy (end)