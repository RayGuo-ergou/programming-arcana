\clearpage
\subsection{Book Overview} % (fold)
\label{sub:book_overview}

This book focuses on programming concepts, and gives you the option of programming these using either the C or Pascal programming language. Both languages are very capable, with each having their own advantages and disadvantages. Pascal was designed as a teaching language and is easy to program with while still being very capable. C is very flexible and is the basis for a number of other languages.

The layout of each chapter is the same, and has the following format:
\begin{enumerate}
  \item \textbf{Concepts}: The first part of the chapter introduces the concepts that will be covered. This is done in a language neutral manner, with the focus being on how to think about the tools being presented. This will introduce each concept with an illustration, and accompany this with explanatory text.
  \item \textbf{Using the Concepts}: The next section shows how these concepts can be used to achieve a task. This task will try to cover all the concepts presented in a practical manner. This is done in a language neutral way, and talks about how to use the concepts to achieve a goal.
  \item \textbf{Languages}: The next two sections present the syntax you need to use these concepts in \textbf{C} and \textbf{Pascal}. You should use this as a reference, and can read this alongside reading about how to use the concepts.
  \item \textbf{Understanding}: Following the language specific details, the next section explains in detail how the concepts work within the computer. Use this to get an understanding of how the concepts work in more detail. This section will show you illustrations of what is happening within the computer when your code is running.
  \item \textbf{Examples}: Each chapter will have at least one example showing you how these concepts can be used. This will include the code, and some explanatory text to discuss what is being presented.
  \item \textbf{Exercises}: The exercises allow you to put into practice what you have read about. You cannot learn to program without practice. These exercises are a good start, but you should try to come up with your own project so that you can test out these new concepts on something you are interested in working on.
\end{enumerate}



\clearpage
\subsubsection{Formatting} % (fold)
\label{ssub:formatting}

This book has a number of visual formatting guides. These are designed to help you navigate through the material easily.

\pseudosection{
Text formatted in this way relates to an algorithm description. This will describe the steps that need to be performed in a way that is language neutral and can be applied to C, Pascal, and possibly other languages.
}

\csection{
Text formatted in this way relates to the C programming language. If you are going to use C you need to pay attention to the text in these boxes, otherwise you can skip over them.
}

\passection{
Text formatted in this way relates to the Pascal programming language. If you are going to use Pascal you need to pay attention to the text in these boxes, otherwise you can skip over them.

}

\mynote{
Text formatted in this way covers notes related to the current concept or illustration. This book makes extensive use of notes to capture important points, so do not skip over these.
}

The language sections of each chapter also add markers to each page to clearly mark where they start, and where they end. If this is your first programming experience you should stick with one of these languages, so you can skip the pages that are marked as being for the other language.


% subsubsection formatting (end)


% subsection book_overview (end)
