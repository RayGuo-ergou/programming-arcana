\subsection{Hello World in C++} % (fold)
\label{sub:hello_world_in_c}

When you follow the SplashKit install instructions, select the C++ language and install the clang++ or g++ compiler.

Once you have the compiler installed, you can create your first program: the famous \textbf{Hello World} discussed in \sref{sub:hello world}. The C++ code for this is shown in \lref{lst:hello-world-c-c}. This code tells the computer to `write' the text \emph{Hello World!} to the Terminal, and follow it with a new `line'. Do the following to create this program for yourself, see the notes below for hints:

\begin{enumerate}
  \item Open a Terminal
  \item Navigate to where you want to save your code, and create a new folder using \texttt{mkdir} and the project name (without spaces).
  \item Move into the project folder using the \texttt{cd} command.
  \item Use \texttt{skm} to create a new C++ project.
  \item Open Visual Studio Code, and open the \textbf{folder} that contains your project code.
  \item Type\footnote{Do not just copy and paste it out of the text, type it in yourself as this will help you learn the concepts being covered.} in the text below, making sure you get every character correct.
  \item Compile the program using \texttt{skm clang++ program.cpp -o HelloWorld}
  \item Run the program using \texttt{./HelloWorld}
\end{enumerate}

Well done, you have now created and run your first C++ program!

\csection
{
\ccode{lst:hello-world-c-c}{Hello World code in C++.}{code/c/program-creation/hello-world.c}
}

\mynote{
\begin{itemize}
  \item See \sref{subs:install} \nameref{subs:install} for details on installing the tools you need.
  \item See \nameref{ssub:bash} in \sref{sub:terminal} for an example of how to use the Terminal.
  \item See \sref{sec:using_these_concepts_compiling_a_program} \nameref{sec:using_these_concepts_compiling_a_program} for the overall process and the output you should expect from the program.
\end{itemize}
}

% subsection hello_world_in_c (end)