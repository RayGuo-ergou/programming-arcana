The C programming language is very powerful and flexible. In this section you will see the tools that you need to install to start programming with C on your computer.

To explore this material you will need a small program

\subsection{Hello World in C} % (fold)
\label{sub:hello_world_in_c}

The program used in the last chapter was the classic Hello World program. The C code for this is shown in \lref{lst:hello-world-c-c}. This code will be explained in future chapters, for the moment you will need to copy\footnote{Do not just copy and paste it out of the text, type it in yourself as this will help you learn the concepts being covered.} it into a source code file, save it, and then compile it.

\csection
{
\ccode{lst:hello-world-c-c}{Hello World code in C.}{code/c/program-creation/hello-world.c}
}

\mynote{
\begin{itemize}
  \item This section will show you the tools you need to install to get started.
  \item The Hello World program is a good start, it outputs text to the Terminal when it is executed.
  \item You will need to install the following tools to type, compile, and run this:
  \begin{enumerate}
    \item The \textbf{gcc compiler} to compile your code.
    \item A \textbf{text editor} to enter your code into.
  \end{enumerate}
  \item This chapter will also show you how to create graphical programs using SwinGame.
\end{itemize}
}

% subsection hello_world_in_c (end)