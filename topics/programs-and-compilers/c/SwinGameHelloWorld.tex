\clearpage
\subsection{Graphical Applications with SwinGame} % (fold)
\label{sub:graphical_applications_with_swingame}

SwinGame is a 2D game creation library. It contains a number of resources that you can use to create small games using the C programming language. To get started with SwinGame you need to download a \emph{template} from the website. The template includes everything you need to get started creating a game in SwinGame.

\subsubsection{Coding a SwinGame} % (fold)
\label{ssub:coding_a_swingame}

The code for your SwinGame can be found in the \textbf{\texttt{src}} folder. This includes a \texttt{\textbf{GameMain.c}} file, where you can place your source code.

% subsubsection coding_a_swingame (end)

\subsubsection{Compiling a SwinGame} % (fold)
\label{ssub:compiling_a_swingame}

There are a number of steps that you need to follow to compile a SwinGame. Fortunately, all of these are written for you in the script called \textbf{\texttt{build.sh}}. This script to included in the project template. 

% subsubsection compiling_a_swingame (end)

\subsubsection{Running a SwinGame} % (fold)
\label{ssub:running_a_swingame}

% subsubsection running_a_swingame (end)

% subsection graphical_applications_with_swingame (end)