\clearpage
\subsection{C Terminal Output} % (fold)
\label{sub:c_console_output}

C comes with a range of libraries that provide reusable programming artefacts, including reusable \nameref{sub:procedure}s. The C library includes a number of different components, one of which is \texttt{stdio.h} - the standard input/output library. This header file gives you access to artefacts you can use to perform input and output tasks, including code to write output to the Terminal. The \texttt{printf} procedure is used to write output data to the Terminal.

\begin{table}[h]
  \centering
  \begin{tabular}{|c|p{9cm}|}
    \hline
    \multicolumn{2}{|c|}{\textbf{Procedure Prototype}} \\
    \hline
    \multicolumn{2}{|c|}{} \\
    \multicolumn{2}{|c|}{\texttt{printf(char *format, \ldots )}} \\
    \multicolumn{2}{|c|}{} \\
    \hline
    \textbf{Parameter} & \textbf{Description} \\
    \hline
    \texttt{ format } & The text that is to be written to the Terminal. This text may contain format tags to include other values. See Figure \ref{csynt:program-creation-format-string} for the syntax of the \textbf{format tag}. \\
    & \\
    \texttt{\ldots}   & Optional values, must have at least as many values as format tags. \\
    \hline
  \end{tabular}
  \caption{Parameters that must be passed to \texttt{printf}}
  \label{tbl:program-creation-c printf parameters}
\end{table}

The syntax for the \emph{format tag} is shown in Figure \ref{csynt:program-creation-format-string}, with the details for the values that can be placed in the \emph{flag}, \emph{width}, \emph{precision}, and \emph{specifier} section being shown in Table \vref{tbl:program-creation-c printf specifier}. A number of examples are shown in Table \ref{tbl:program-creation-c printf specifier}, as well as in Listing \ref{lst:program-creation-c-printf}.

\csyntax{csynt:program-creation-format-string}{Format Tag}{program-creation/format-string}

\csection{\ccode{lst:program-creation-c-printf}{C \texttt{printf} examples}{code/c/program-creation/sample-printf.c}}

\begin{table}[htbp]
  \begin{minipage}{\textwidth}
  \centering
  
  \begin{tabular}{|c|p{4cm}|l|c|}
    \hline
    \textbf{Flag} & \textbf{Description}  & \multicolumn{2}{c|}{ \textbf{Example Usage \& Output} } \\
    \hline
    \texttt{-}  & Left justify\footnote{Right justify is the default} the width. & \csnipet{printf("\%-5c", 'a');} & \texttt{a\textvisiblespace\textvisiblespace\textvisiblespace\textvisiblespace} \\
                & & \csnipet{printf("\%5i", 23);} & \texttt{\textvisiblespace\textvisiblespace{\textvisiblespace}23} \\
    \hline
    \texttt{+}  & Always display sign for numbers & \csnipet{printf("\%+d", 42);} & \texttt{+42} \\
                & & \csnipet{printf("\%+i", -42);} & \texttt{-42} \\
    \hline
    \texttt{\textvisiblespace}\footnote{A space.}  & Shows a space if positive & \csnipet{printf("\% f", 127.5);} & \texttt{\textvisiblespace127.5} \\
                & & \csnipet{printf("\% i", -73);} & \texttt{-73} \\
    \hline
    \texttt{0}  & Pad with 0's rather than spaces\footnote{The 5 in the example represents the width of the output.} & \csnipet{printf("\%05i", 3);} & \texttt{00003} \\
    \hline
    \multicolumn{4}{c}{} \\
    \hline
    \textbf{Width} & \textbf{Description}  & \multicolumn{2}{c|}{ \textbf{Example Usage \& Output} } \\
    \hline
    \emph{number} & The minimum width for output & \csnipet{printf("\%5i", 1);} & \texttt{\textvisiblespace\textvisiblespace\textvisiblespace{\textvisiblespace}1} \\
    & & \csnipet{printf("\%5s", "Fred");} & \texttt{{\textvisiblespace}Fred} \\
    & & \csnipet{printf("\%5s", "Hello World");} & \texttt{Hello World}\footnote{Width specifies the minimum width.} \\
    \hline
    \multicolumn{4}{c}{} \\
    \hline
    \textbf{Precision} & \textbf{Description}  & \multicolumn{2}{c|}{ \textbf{Example Usage \& Output} } \\
    \hline
    \emph{number} & \textbf{For integers}: same as Width &  \csnipet{printf("\%.5i", 1);} & \texttt{\textvisiblespace\textvisiblespace\textvisiblespace{\textvisiblespace}1} \\
     & \textbf{For real numbers}: number of &  \csnipet{printf("\%.3f", 3.1415);} & \texttt{3.142} \\
     &  values after the decimal point & \csnipet{printf("\%.3f", 2.5);} & \texttt{2.500} \\
    \hline
    \multicolumn{4}{c}{} \\
    \hline
    \textbf{Specifier} & \textbf{Output}  & \multicolumn{2}{c|}{ \textbf{Example Usage \& Output} } \\
    \hline
    \texttt{c}  & A single character & \csnipet{printf("\%c", 'a');} & \texttt{a} \\
    \hline
    \texttt{d} or \texttt{i} & A signed decimal integer & \csnipet{printf("\%d", -127);} & \texttt{-127} \\
    \hline
    \texttt{f}  & Decimal floating point number & \csnipet{printf("\%f", 127.5);} & \texttt{127.5} \\
    \hline
    \texttt{s}  & Text data & \csnipet{printf("\%s", "Hello World");} & \texttt{Hello World} \\
    \hline
    \multicolumn{4}{c}{} \\
    \hline
  \end{tabular}
  
  \end{minipage}
  \caption{Details for, and examples of, the format tag specifier, flag, precision, and width.}
  \label{tbl:program-creation-c printf specifier}
\end{table}





% subsection c_console_output (end)