Section \vref{sec:using_these_concepts_program_creation} of this chapter introduced an `Output Test' program, and its design. The pseudocode from this section is shown in Listing \ref{lst:program-creation-hello-pseudo 1}. In this Section you will see the rules for translating this program's design into the C code shown in Listing \ref{lst:program-c-output_test}.

\pseudocode{lst:program-creation-hello-pseudo 1}{Pseudocode for Hello World program (from Listing \ref{lst:program-creation-hello-pseudo}).}{./topics/program-creation/application/HelloWorld.txt}

\csection{\ccode{lst:program-c-output_test}{Output Test in C}{code/c/program-creation/output_test.c}}

\mynote {
\begin{itemize}
  \item Save the C code in a file named \texttt{output\_test.c}.
  \item Compile this using \bashsnipet{gcc -o:OutputTest output_test.c}.
  \item Run using \bashsnipet{./OutputTest}.
  \item The code at the start is a Comment describing what is in the file, see \nameref{sub:c_comments}.
  \item The code \csnipet{#include <stdio.h>}, is part of the \nameref{sub:program_in_c}. It is a \emph{header include}, and gives access to the code in the \emph{Standard IO} Library.
  \item \csnipet{int main() { ... }} is part of the \nameref{sub:program_in_c}, it marks the entry point and contains the instructions that are executed when the program runs.
  \item The \texttt{main} function contains three \nameref{sub:program-creation-c_procedure_call}s. Each is a call to the \texttt{printf}. \nameref{sub:procedure}, which is used to output text to the console. See \nameref{sub:c_console_output}.
  \item Each of the procedure calls contains one or two \nameref{sub:expression}s that pass values to the \texttt{printf} procedure, which will output these to the Terminal.
  \item This code uses \texttt{c-strings}, \texttt{int}, and \texttt{float} types, see \nameref{sub:program-creation-c_types}.
\end{itemize}
}