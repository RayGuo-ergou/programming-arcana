\clearpage
\subsection{C Identifier} % (fold)
\label{sub:c_identifier}

The C \nameref{sec:program-creation-identifier} syntax is shown in Figure \ref{csynt:program-creation-identifier}. In C, as in most languages, the identifier must start with an underscore (\_) or a letter, in other words your identifiers cannot start with a number or contain other symbols. This is because the compiler needs a way of distinguishing identifiers from numbers entered in the code as Literals.

\csyntax{csynt:program-creation-identifier}{an Identifier}{program-creation/identifier}

\begin{table}[h]
  \centering
  \begin{tabular}{|ccccc||cc|}
    \hline
    \multicolumn{5}{|c||}{\textbf{Keywords}} & \multicolumn{2}{c|}{\textbf{Example Identifiers}} \\
    \hline
    \texttt{auto}     &   \texttt{break}    & \texttt{case}     &   \texttt{char}     &   \texttt{const}   & printf & scanf  \\         
    \texttt{continue} &   \texttt{default}  &  \texttt{do}      &   \texttt{double}   &   \texttt{else}    & bitmap & sound\_effect  \\
    \texttt{enum}     &   \texttt{extern}   & \texttt{float}    &   \texttt{for}      &   \texttt{goto}    & name & draw\_bitmap  \\
    \texttt{if}       &   \texttt{int}      &   \texttt{long}   &   \texttt{register} &   \texttt{return}  & age & my\_alien \\         
    \texttt{short}    &   \texttt{signed}   & \texttt{sizeof}   &   \texttt{static}   &   \texttt{struct}  & height & test  \\          
    \texttt{switch}   & \texttt{typedef}  &   \texttt{union}    &   \texttt{unsigned} &   \texttt{void}    & alien & name3 \\
    \texttt{volatile} &   \texttt{while}    &  & &                                                         & \_23  & i \\
    \hline
  \end{tabular}
  \caption{C Keywords and example Identifiers}
  \label{tbl:program-creation-c identifiers and keywords}
\end{table}












\mynote{
\begin{itemize}
  \item Table \ref{tbl:program-creation-c identifiers and keywords} contains a list of the keywords in C, and some example identifiers.
  \item Each item in Table \ref{tbl:program-creation-c identifiers and keywords} is a valid identifier.
  \item A letter is any alphabetic character (\emph{a} to \emph{z} and \emph{A} to \emph{Z}).
  \item A digit is a single number (\emph{0} to \emph{9}).
  \item A \textbf{keyword} is a kind of identifier that has special meaning to the language. Usually used to identify a kind of action, or a kind of artefact.
  \item Notice in the syntax definition that Identifiers cannot contain spaces, or special characters other than underscores (\_).
\end{itemize}
}

% subsection c_identifier (end)