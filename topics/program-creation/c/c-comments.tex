\clearpage
\subsection{C++ Comments} % (fold)
\label{sub:c_comments}

Comments allow you to embed documentation and explanatory text within your program's code. The comments are skipped by the compiler, so they have no affect on the program's machine code. You write comments to help yourself and other people understand what you intend the program to do, and any thoughts you want to record along with the code.

\csyntax{csynt:program-creation-comment}{comments}{program-creation/comment}

\mynote {
\begin{itemize}
  \item Figure \ref{csynt:program-creation-comment} shows the syntax for comments in C++.
  \item In standard C++ the first style of comments must be used, \csnipet{/* Comment */}.
  \item Most modern C++ compilers also allow single line comments using \csnipet{// Comment}.
  \item Standard C++ comments can span multiple lines, these are also known as `\emph{block comments}'.
  \item A compiler ignores comments when compiling your code.
  \item You can type almost anything in the comment, represented by the \texttt{...} in the diagram.
\end{itemize}
}

% subsection c_comments (end)