\clearpage
\subsection{C Expression} % (fold)
\label{sub:program-creation-c_expression}

An \nameref{sub:expression} in C is a mathematical calculation or a Literal value. Each expression will have a \nameref{sub:type}, and can contain a number of mathematic operators. Table \ref{tbl:program-creation-c operators and expresions} lists the operators that you can include in your expressions, listed in order of precedence.\footnote{Expressions follow the standard mathematic order of precedence (BODMAS).} The operators you can use depend on the kind of data that you are using within the expression.

\begin{table}[h]
  \begin{minipage}{\textwidth}
  \centering
  \begin{tabular}{|c|l|l|}
    \hline
    \textbf{Operator} & \textbf{Description} & \textbf{Example} \\
    \hline
    \texttt{ ( ) }     &   Parenthesis                 & \texttt{(1 + 1) * 2}  \\
    \texttt{\% * /}      &   Modulo\footnote{The remainder after division. For example 9 modulo 3 is 0, 10 modulo 3 is 1, 11 modulo 3 is 2 etc.}, Multiplication and Division & \texttt{1 / 2 * 5 \% 3}    \\
    \texttt{+ -}      &   Addition and subtraction    & \texttt{10 + 3 - 4}   \\
    \hline
  \end{tabular}
  \end{minipage}
  \caption{C Operators and Example Expressions}
  \label{tbl:program-creation-c operators and expresions}
\end{table}

\begin{table}[h]
  \begin{minipage}{\textwidth}
  \centering
  \begin{tabular}{|c|c|l|}
    \hline
    \textbf{Example Expression} & \textbf{Value} & \textbf{Type} \\
    \hline
    \texttt{ 73 }     &   73                 & \texttt{int}  \\
    \texttt{ 2.1 }      & 2.1   & \texttt{float}    \\
    \texttt{ "Hello World" }      &   "Hello World"    & \texttt{char*}   \\
    \texttt{ "Fred" }      &   "Fred"    & \texttt{char*}   \\
    \texttt{ 3 * 2 } & 6 & \texttt{int} \\
    \texttt{ 1 + 3 * 2 }  & 7 & \texttt{int} \\
    \texttt{ (1 + 3) * 2} & 8 & \texttt{int} \\
    \texttt{ 7 - 3 + 1 }  & 5 & \texttt{int} \\
    \texttt{ 3 / 2 } & 1\footnote{C does integer division for int values, rounding the value down.} & \texttt{int} \\
    \texttt{ 3.0 / 2.0} & 1.5 & \texttt{float} \\
    \texttt{ 3 \% 2} & 1 & \texttt{int} \\
    \texttt{ 11 \% 3} & 2 & \texttt{int} \\
    \texttt{ 3 / 2.0 } & 1.5\footnote{If either, or both, values are real (floating point) numbers the result is also a real number.} & \texttt{float} \\
    \texttt{ 1 + (3 / 2.0) + 6 * 2 - 8} & 6.5 & \texttt{float} \\
    \hline
  \end{tabular}
\end{minipage}
  \caption{C Operators and Example Expressions}
  \label{tbl:program-creation-c example expresions}
\end{table}


\mynote{
\begin{itemize}
  \item Table \ref{tbl:program-creation-c example expresions} shows some example expressions, their values, and types.
  \item Expressions can be literal values, entered in the code.
  \item Expression can contain mathematical calculations using standard addition, subtraction, multiplication, division, and grouping.
\end{itemize}
}


% subsection c_expression (end)