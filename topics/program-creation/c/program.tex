\clearpage
\subsection{C Program} % (fold)
\label{sub:program_in_c}

The C programming language does not have an explicit Program artefact for you to create. Rather, in C a program is implied by the existence of a special function called `\texttt{main}' somewhere in your source code. Figure \ref{csynt:program-creation-program} shows the structure of the syntax you can use to create a program using the C language.

\csyntax{csynt:program-creation-program}{a Program}{program-creation/program}

The code in Listing \ref{lst:program-creation-c-hello-world} shows an example C Program. You should be able to match this up with the syntax defined in Figure \ref{csynt:program-creation-program}. Notice at the start of the code the syntax indicates we can have an optional \emph{header include}, this matches up with the first line in the code where it \emph{includes} the `stdio.h'\footnote{Read `stdio.h' as `standard IO'. This is the file that gives you access to a \nameref{sub:library} that contains procedures you can use for standard (std) input and output (IO).} header file. Declaration of the \texttt{main} function follows the inclusion of the header file, and it contains the instructions that are executed when the program runs.

\csection{\ccode{lst:program-creation-c-hello-world}{C Hello World}{code/c/program-creation/hello-world.c}}

\mynote{
  \begin{itemize}
    \item A C \nameref{sub:program} starts at the first \nameref{sub:statement} within the \texttt{main} function (line 5).
    \item A function is a kind of \nameref{sub:procedure}, and their details will be covered later.
    \item The `\texttt{return 0}' code is a \nameref{sub:statement} that ends the \texttt{main} function (and the program).
    \item With the \emph{header include} syntax you use \csnipet{#include <...>} to include standard libraries, and \csnipet{#include "..."} to include other external libraries.
  \end{itemize}
}

% subsection program_in_c (end)