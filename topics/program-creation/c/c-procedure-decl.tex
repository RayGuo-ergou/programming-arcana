\clearpage
\subsection{C++ Procedure Declaration} % (fold)
\label{sub:c_procedure_declaration}

The Syntax for a C++ Procedure Declaration is shown in Figure \ref{csynt:procedure-decl-procedure-decl}.

\csyntax{csynt:procedure-decl-procedure-decl}{a Procedure}{procedure-decl/procedure-decl}

\csection{\ccode{lst:program-c-print-steps}{Cooking a Meal}{code/c/procedure-decl/print-steps.c}}

\mynote{
\begin{itemize}
  \item There are three Procedures declared in the code in Listing \ref{lst:program-c-print-steps}.
  \item A \textbf{Procedure Declaration} starts with the word \textbf{\texttt{void}}. This indicates that the following code is a procedure declaration to the compiler.
  \item The \textbf{Procedure Name} is an identifier. It is the name of the Procedure. This can be any valid \nameref{sub:c_identifier} that has not been used before.
  \item The empty parenthesis must appear after the procedure's name, and before the \emph{block}.
  \item The \textbf{block} should look familiar. This is the same as was used in the \emph{main function} of the program to define its instructions, and is used for the same purpose within the \emph{Procedure Declaration}.
  \bigskip
  \item There are a number of conventions, called coding standards, that describe how your code should appear for a given language. In this text we will use a common C convention of having all \emph{Procedure Names} in \textbf{lower case}, with underscores ( \_ ) used to separate words. So the \emph{Get Ingredients} procedure becomes \texttt{get\_ingregients}.
\end{itemize}
}

% subsection c_procedure_declaration (end)