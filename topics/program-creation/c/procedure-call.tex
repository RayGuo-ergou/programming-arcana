\clearpage
\subsection{C Procedure Call} % (fold)
\label{sub:program-creation-c_procedure_call}

A procedure call allows you to run the code in a Procedure, getting its instructions to run before control returns back to this point in the program.

\csyntax{csynt:program-creation-procedure-call}{Procedure Call Syntax}{program-creation/procedure-call}

\csection{\ccode{lst:program-creation-c-count-back}{C Count Back}{code/c/program-creation/count-back.c}}

\mynote{
\begin{itemize}
  \item A Procedure Call is an \textbf{action}, it commands the computer to run the code in a Procedure.
  \item The Procedure Call starts with the Procedure's \nameref{sub:identifier}, this indicates the procedure to be called.
  \item Following the Identifier is a list of values within parenthesis,  these are the values (coded as \nameref{sub:expression}s) that are passed to the procedure for it to use.
  \item Remember that C is case sensitive so using \texttt{Printf} instead of \texttt{printf} will not work.
  \item The code in Listing \ref{lst:program-creation-c-count-back} contains a \nameref{sub:program_in_c}.
  \item This Program contains four procedure calls.
  \item Each procedure call runs the \texttt{printf} procedure to output text to the Terminal. See the section on \nameref{sub:c_console_output}.
  
\end{itemize}
}


% subsection c_procedure_call (end)