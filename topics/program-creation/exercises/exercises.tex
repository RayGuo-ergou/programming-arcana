Read over the concepts in this chapter and answer the following questions:
\begin{enumerate}
  \item What is a \nameref{sub:program}? What does it contain?
  \item A program is an artefact, something you can create in code. Why would you want to create a program your code?
  \item What is the entry point of a program? Why is it important to know where this is in your code?
  \item Do you have to write all of the code for your program, or are you able to use artefacts from elsewhere?
  \item What is a \nameref{sub:statement}?
  \item What statement was introduced in this chapter?
  \item What is a \nameref{sub:procedure}?
  \item Where can you find procedures you may be interested in using?
  \item What is an \nameref{sub:expression}?
  \item Where are expressions coded? Give an example showing an expression being used in code.
  \item What are the three broad kinds of \nameref{sub:type}s that a language will provide?
  \item What are the names of the types in the language you are using? Name the main type you are likely to work with for each of the broad kinds of types.
  \item What are the values of the following expressions? Which types could these values be used with (possibly multiple)? Note: some answers are dependent of the language you are using.
  \begin{multicols}{3}
  \begin{enumerate}
    \item 5
    \item 3 + 5 * 2
    \item (3 + 5) * 2
    \item 3.1415
    \item 1 / 2 
    \item 1.0 / 2.0
    \item 2 + 1 / 2.0 * 6
    \item "Fred Smith" (C)
    \item 'Fred Smith' (Pascal)
  \end{enumerate} 
  \end{multicols}
  \item Where can statements be coded in your program?
  \item What is the role of an \nameref{sub:identifier}? What artefacts be be identified?
  \item What is a keyword?
  \item Which of the following are valid identifiers?
  \begin{multicols}{4}
    \ttfamily
    \begin{enumerate}
      \item hello
      \item \_123
      \item fred
      \item my name
      \item my\_name
      \item begin
      \item void
      \item Main
      \item 1234
      \item a1
      \item 3.1415
      \item WOW\_COOL
    \end{enumerate}
  \end{multicols}
  \item What is a \nameref{sub:library}, and what does it contain?
  \item What happens to \nameref{sub:comments} when you code is compiled? Why do languages include these? Why are they considered important by good developers?
  \item What is the stack? What does it keep track of when your program is running?
  \item What is loaded into memory when your program is started?
  \item What happens in the computer when a procedure is called?
\end{enumerate}

\clearpage
Apply what you have learnt to the following tasks:

\begin{enumerate}
  \item Write a program that prints the 5 times table from 1 * 5 to 10 * 5. See \tref{tbl:five-times}.
  \begin{itemize}
    \item Think about the artefacts you will create, and use.
    \item Write pseudocode for the program's instructions
    \item Convert your pseudocode to either C or Pascal
    \item Compile and Run your program, and check that the values are correctly calculated
  \end{itemize}
  
  \begin{table}[h]
  \centering
  \begin{tabular}{l|p{12cm}}
    \hline
    \multicolumn{2}{c}{\textbf{Program Description}} \\
    \hline
    \textbf{Name} & \emph{Five Times Table} \\
    \\
    \textbf{Description} & Displays the 5 times table from $1 \times 5$ to $10 \times 5$. \\
    \hline
  \end{tabular}
  \caption{Description of the \emph{Five Times Table} program.}
  \label{tbl:five-times}
  \end{table}
  
  
  \item Write a program that prints the powers\footnote{In the code you will need to calculate these manually using times ($2^1$ = 2, $2^2$ = 2*2, $2^3$ = 2*2*2, etc.)} of 2 from $2^1$ to $2^8$. See \tref{tbl:two-powers}.
  \begin{itemize}
    \item Think about the artefacts you will create, and use.
    \item Write pseudocode for the program's instructions
    \item Convert your pseudocode to either C or Pascal
    \item Compile and Run your program, and check that the values are correctly calculated
  \end{itemize}
  
  \begin{table}[h]
  \centering
  \begin{tabular}{l|p{12cm}}
    \hline
    \multicolumn{2}{c}{\textbf{Program Description}} \\
    \hline
    \textbf{Name} & \emph{Powers of Two} \\
    \\
    \textbf{Description} & Displays the powers of 2 from $2^{1}$ to $2^{8}$. \\
    \hline
  \end{tabular}
  \caption{Description of the \emph{Five Times Table} program.}
  \label{tbl:two-powers}
  \end{table}
  
  
  \item Write a program that prints the 73 times table from 1 * 73 to 10 * 73. See \tref{tbl:sevelty-three-times}.
  \begin{itemize}
    \item Think about the artefacts you will create, and use.
    \item Write pseudocode for the program's instructions
    \item Convert your pseudocode to either C or Pascal
    \item Compile and Run your program, and check that the values are correctly calculated
  \end{itemize}
  
  \begin{table}[h]
  \centering
  \begin{tabular}{l|p{12cm}}
    \hline
    \multicolumn{2}{c}{\textbf{Program Description}} \\
    \hline
    \textbf{Name} & \emph{Seventy Three Times Table} \\
    \\
    \textbf{Description} & Displays the 73 times table from $1 \times 73$ to $10 \times 73$. \\
    \hline
  \end{tabular}
  \caption{Description of the \emph{Seventy Three Times Table} program.}
  \label{tbl:sevelty-three-times}
  \end{table}
  
  \clearpage
  \item Write a program that prints a table showing calculations of circle dimensions. This should output the radius, circle area, diameter, and circumference of circles with a radius of 1cm, 1.5cm, and 2cm. See \tref{tbl:circle-dimensions}.
  \begin{itemize}
    \item Find the necessary calculations and think about the artefacts you will use and create.
    \item Write pseudocode for the program's instructions.
    \item Convert your pseudocode to either C or Pascal.
    \item Compile and Run your program, and check that the values are correctly calculated.
  \end{itemize}
  
  \begin{table}[h]
  \centering
  \begin{tabular}{l|p{12cm}}
    \hline
    \multicolumn{2}{c}{\textbf{Program Description}} \\
    \hline
    \textbf{Name} & \emph{Circle Dimensions} \\
    \\
    \textbf{Description} & Displays a table of circle dimensions for circles with a radius of 1cm, 1.5cm, and 2cm. This will output the radius, circle area, diameter, and circumference of circles. \\
    \hline
  \end{tabular}
  \caption{Description of the \emph{Circle Dimensions} program.}
  \label{tbl:circle-dimensions}
  \end{table}
  
  \item Write a program with SwinGame that draws a face using primitive shapes. See \tref{tbl:face-shape}.
  \begin{itemize}
    \item Draw up an outline of the program. Work out the coordinates of the circles for the face, circles. Determine three points of the triangle for the mouth.
    \item Write up the pseudocode for the program's instructions. Remember to use the \texttt{RefreshScreen} and \texttt{Delay} procedures to see the results.
    \item Convert your pseudocode to either C or Pascal
    \item Compile and Run your program, and check that the values are correctly calculated
  \end{itemize}
  
  \begin{table}[h]
  \centering
  \begin{tabular}{l|p{12cm}}
    \hline
    \multicolumn{2}{c}{\textbf{Program Description}} \\
    \hline
    \textbf{Name} & \emph{Face Shape} \\
    \\
    \textbf{Description} & Displays face to the screen using SwinGame. \\
    \hline
  \end{tabular}
  \caption{Description of the \emph{Face Shape} program.}
  \label{tbl:face-shape}
  \end{table}
  
\end{enumerate}

\bigskip
If you want to further your knowledge in this area you can try to answer the following questions. The answers to these questions will require you to think harder, and possibly look at other sources of information.
\begin{enumerate}
  \item C and Pascal are imperative programming languages. What does this mean, and how does it relate to the way you think about code?
  \item Artefacts are things that you can create. What artefacts where introduced in this chapter? If you could create these artefacts, what would you do with them?
\end{enumerate}
