\clearpage
\section{Pascal Program} % (fold)
\label{sec:pascal-basic-program}

Pascal includes syntax to allow you to declare a program in your code. The code for a Pascal program is written

Your program contains the instructions that are followed when the user runs the executable file. In many cases you will want to use code from other \nameref{sec:unit}s. This is achieved using the uses clause. By default all program's have access to the \texttt{System} unit, which contains \nameref{sec:writeln} and many other useful procedures.

\subsection*{Pascal Syntax} % (fold)
\label{sub:programsgeneralform}

% \clearpage
% \subsection{C++ Program} % (fold)
% \label{sub:program_in_c}

% The C++ programming language does not have an explicit Program artefact for you to create. Rather, in C++ a program is implied by the existence of a special function called `\texttt{main}' somewhere in your source code. Figure \ref{csynt:program-creation-program} shows the structure of the syntax you can use to create a program using the C++ language.

% \csyntax{csynt:program-creation-program}{a Program}{program-creation/program}

% The code in Listing \ref{lst:program-creation-c-hello-world} shows an example C++ Program. You should be able to match this up with the syntax defined in Figure \ref{csynt:program-creation-program}. Notice at the start of the code the syntax indicates we can have an optional \emph{header include}, this matches up with the first line in the code where it \emph{includes} the `splashkit.h' header file. Declaration of the \texttt{main} function follows the inclusion of the header file, and it contains the instructions that are executed when the program runs.

% \csection{\ccode{lst:program-creation-c-hello-world}{C++ Hello World}{code/c/program-creation/hello-world.c}}

% \mynote{
%   \begin{itemize}
%     \item When a C++ \nameref{sub:program} runs, it start running the instructions from the first \nameref{sub:statement} within the \texttt{main} function (line 5).
%     \item A \nameref{sub:function} is a kind of \nameref{sub:procedure}, and their details will be covered later (see \sref{sub:function}).
%     \item The `\texttt{return 0}' code is a \nameref{sub:statement} that ends the \texttt{main} function (and the program). The \nameref{sub:return_statement} is covered later in \sref{sub:return_statement}.
%     \item With the \emph{header include} syntax you use \csnipet{#include <...>} to include standard libraries, and \csnipet{#include "..."} to include other external libraries.
%     \item Header files contain a summary of the features available within a library. By including the header file you gain access to these features.
%   \end{itemize}
% }

% % subsection program_in_c (end)
% subsection general_form (end)

\subsection*{Example} % (fold)
\label{sub:programExamples}

The code in Listing \ref{lst:pascal-program-creation-HelloWorld} and Listing \ref{lst:pascal-program-creation-ClrScr} show two example programs. Listing \ref{lst:pascal-program-creation-HelloWorld} contains the instructions that for a program which greets the world when it is executed, printing the text \emph{Hello World} to the console. The program in Listing \ref{lst:pascal-program-creation-ClrScr} shows how to use code from the CRT \nameref{sec:unit}, the code for the \texttt{ClrScr} procedure is contained in the \texttt{CRT} Unit.

\lstinputlisting[caption={The classic ``Hello World'' program.},label={lst:pascal-program-creation-HelloWorld}]{./topics/program-creation/pascal/HelloWorld.pas}
\lstinputlisting[caption={Example of unit use, using the \texttt{CRT} unit},label={lst:pascal-program-creation-ClrScr}]{code/pascal/program-creation/ClearScreen.pas}

% subsection program_examples (end)

\mynote{
\begin{itemize}
    \item The program in Listing~\ref{lst:HelloWorld} is named `HelloWorld', see \nameref{sec:identifier}.
    \item A program's instructions start at the \texttt{begin} \nameref{sec:keyword}.
    \item The `Hello World' program in Listing~\ref{lst:HelloWorld} has one statement that uses \nameref{sec:writeln}.
    \item The statements are grouped within a \emph{block} that ends with the \texttt{end} \nameref{sec:keyword}.
    \item Program's end at the full stop (at \texttt{end.}).
    \item \nameref{sec:writeln} comes from the System unit, to use code in other units you need to include the uses clause.
\end{itemize}
}
% subsection program_study (end)

% section program (end)