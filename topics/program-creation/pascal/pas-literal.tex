\clearpage
\subsection{Pascal Literal} % (fold)
\label{sub:program-creation-pas_literal}

A literal is either a number or text value stated directly in the code. In other words, it is not \emph{calculated} when the program runs - it is already in the code. Figure \ref{passynt:program-creation-literal} shows the syntax for the different literal values you can enter into your Pascal code.

\passyntax{passynt:program-creation-literal}{Literals}{program-creation/literal}

\mynote{
\begin{itemize}
  \item `0..9' means the digits 0, 1, 2, etc. up to 9.
  \item To embed a single quote (') in a string you need to use two single quotes, for example \texttt{WriteLn('It''s a lovely day!');}.
\end{itemize}
}

% subsection c_literal (end)