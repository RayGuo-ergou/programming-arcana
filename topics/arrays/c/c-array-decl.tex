\clearpage
\subsection{C Array Declaration} % (fold)
\label{sub:c_array_declaration}

C allows you to declare variables that are arrays. This is done using the \texttt{[ ]} to denote the number of elements in the array (\emph{n}). Indexes will then be \emph{0} to \emph{n-1}.

\csyntax{csynt:array-decl}{Array Variable and Type Declarations}{arrays/array-decl}

\csection{\ccode{clst:test-array}{C code demonstrating array declaration}{code/c/array/test-array.c}}

\mynote{
\begin{itemize}
  \item This is the C syntax to declare a \nameref{sub:array}.
  \item Arrays in C do not remember their length, you must keep track of this yourself.
  \item You can initialise an array when it is declared using a list of values in braces (\{\ldots\}). This can only be done to initialise arrays, and is not valid elsewhere.
  \item The size of the array must be able to be determined at compile time.
\end{itemize}
}

% subsection c_array_declaration (end)