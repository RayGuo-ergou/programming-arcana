\section*{Book Overview} % (fold)
\label{sub:book_overview}

This book is designed for people who want to gain an in depth knowledge of software development. It assumes that you are familiar with computers and how to use them, but does require you to have any prior programming experience. You should feel comfortable using a computer, and be prepared to start looking at how it works in more detail.

There are a number of different ways in which texts can present their material. In this book, the content is presented using a concept first approach. This approach places the programming concepts as the central focus, with details of syntax being secondary. The logic behind this is that you can learn new syntax if you understand the concepts, whereas it will be difficult to be productive if you understand the syntax but do not understand the concepts behind this syntax. To this end, this book gives you the choice to study either the C or Pascal programming language. Both languages are very capable, and offer different advantages and challenges.

The chapters of this book build upon each other. Each chapter covers a new group of concepts, that will expand your programming capabilities and enable you to create larger and more capable programs. The layout of each chapter is the same, with the concepts having the main focus. Each chapter is laid out in the following order:
\begin{enumerate}
  \item \textbf{Concepts}: The first part of the chapter introduces the concepts that will be covered. This is done in a language neutral manner, with the focus being on how to think about the tools being presented. This will introduce each concept with an illustration, and accompany this with explanatory text.
  \item \textbf{Using the Concepts}: The next section shows how these concepts can be used to achieve a task. This task will try to cover all the concepts presented in a practical manner. This is done in a language neutral way, and talks about how to use the concepts to achieve a goal.
  \item \textbf{Languages}: The next two sections present the syntax you need to use these concepts in \textbf{C} and \textbf{Pascal}. You should use this as a reference, and can read this alongside reading about how to use the concepts.
  \item \textbf{Understanding}: Following the language specific details, the next section explains in detail how the concepts work within the computer. Use this to get an understanding of how the concepts work in more detail. This section will show you illustrations of what is happening within the computer when your code is running.
  \item \textbf{Examples}: Each chapter will have at least one example showing you how these concepts can be used. This will include the code, and some explanatory text to discuss what is being presented.
  \item \textbf{Exercises}: The exercises allow you to put into practice what you have read about. You cannot learn to program without practice. These exercises are a good start, but you should try to come up with your own project so that you can test out these new concepts on something you are interested in working on.
\end{enumerate}

\section*{Which Language?} % (fold)
\label{sec:which_language_}

A programming language defines a set of rules that determine how you write the code for your programs. Each language defines its own rules, and so there is always the temptation to focus heavily on these details and place the overriding concepts in second place. We believe that when you are starting to learn to program, a good understanding of the programming concepts is far more important than the details of the programming language you are using. This book is not an in depth study of either the C or Pascal language, it is a book about learning to program.

To really learn these concepts well you will need to practice putting them to use. This will require you to use a programming language. Each chapter will provide you with enough information to put the concepts to use in either the C or Pascal language. So the main question you need to answer now is which language will you use?

Both C and Pascal are very capable languages. Pascal was designed as a teaching language, which means that it does make it a little easier to see how the concepts being covered apply to your code. C, on the other hand, is a commercial language designed for professionals to build programs. This is both an advantage and disadvantage for C. The advantage is that the language is widely used in industry, but the disadvantage is that it lacks the clarity that is offered by Pascal. Remember that this is only your first programming language. A professional software developer will know many different languages, and by the end of this material you will be equipped to learn many new languages.

\csection{
C is a very flexible language that is widely used in industry, though its syntax is more of a challenge to learn.
}

\passection{
Pascal is a powerful languages that was designed to teach programming. It is used in industry, but not to the same extent as C.
}

% section which_language_ (end)

\clearpage
\section*{Formatting} % (fold)
\label{sub:formatting}

This book has a number of visual formatting guides. These are designed to help you navigate through the material easily.

\pseudosection{
Text formatted in this way relates to an algorithm description. This will describe the steps that need to be performed in a way that is language neutral and can be applied to C, Pascal, and possibly other languages.
}

\csection{
Text formatted in this way relates to the C programming language. If you are going to use C you need to pay attention to the text in these boxes, otherwise you can skip over them.
}

\passection{
Text formatted in this way relates to the Pascal programming language. If you are going to use Pascal you need to pay attention to the text in these boxes, otherwise you can skip over them.

}

\mynote{
Text formatted in this way covers notes related to the current concept or illustration. This book makes extensive use of notes to capture important points, so do not skip over these.
}

The language sections of each chapter also add markers to each page to clearly mark where they start, and where they end. If this is your first programming experience you should stick with one of these languages, so you can skip the pages that are marked as being for the other language.


% subsubsection formatting (end)


% subsection book_overview (end)
