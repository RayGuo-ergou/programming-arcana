\clearpage
\subsection{Pascal Record Declaration} % (fold)
\label{sub:pas_structure_declaration}

A record is a type that allows you to store multiple field values. In Pascal the record defines a list of fields and their types. The field declarations are similar to other variable declarations, with you specifying the name and then the type of the field.

\passyntax{psynt:record-decl}{Record Declarations}{type-decl/record-decl}

\begin{figure}
  \passection{\pascode{plst:test-struct}{Pascal for working with a record}{code/pascal/types/TestRecord.pas}}
\end{figure}

\mynote{
\begin{itemize}
  \item This is the syntax for declaring your own custom record.
  \bigskip
  \item The declaration starts with \texttt{\textbf{typedef}}, which indicates that this is a declaration for a custom type, then \texttt{\textbf{struct}} indicating the declaration of a structured record.
  \item Next comes an \emph{option} \textbf{struct name}. This identifier can be used to refer to the struct\footnote{This enables you to declare a struct outside of a \texttt{typedef}.} but requires the keyword \texttt{struct} before it. For example, the declaration in \lref{clst:test-struct} declares a \texttt{person}, or a \texttt{struct person\_struct}, depending on if you use the \emph{struct name} or the \emph{typedef name}.
  \item Following this is a \textbf{list of fields} between braces (i.e. \texttt{\{\ldots\}}). Each field has its own type that may be of any type, including other structures, arrays, standard types, enumerations, and unions.
  \item Finally the typedef ends with the \textbf{name} of the type you are declaring.
  \bigskip
  \item \lref{clst:test-struct} shows an example of a record in C. The \texttt{person} record contains an array of fifty characters called \texttt{name}, and an integer called \texttt{age}.
  \item Remember that the type declaration is creating a new type. After declaring the \texttt{struct} you can now create variables of the \texttt{person} type.
  \item Additional fields could be added to the record, and these will be added to all variables declared from this type.
  
\end{itemize}
}

\clearpage

\subsubsection{Pascal Variant Records (Unions)} % (fold)
\label{ssub:pascal_variant_records_unions_}

Pascal records can include a \emph{variant} part, which stores a single value from a list of field options. The variant part comes at the end of the record, starting with the \texttt{case} keyword. The variant parts is matched with a ordinal type (e.g. enumeration) that can also be stored as a \emph{tag} field, indicating which field option is currently storing a value. See \lref{plst:test-union} for an example.

\begin{figure}[h]
  \passection{\pascode{plst:test-union}{Pascal variant record (union)}{code/pascal/types/TestUnion.pas}}
\end{figure}

\mynote{
\begin{itemize}
  \item The variant part of the record stores a \textbf{single} value.
  \item \texttt{MyUnionType} stores three values: \texttt{otherField}, \texttt{kind}, and one value from the variant part.
  \item The \texttt{kind} field of the \texttt{MyUnionType} record indicates which of the three field options is storing a value. This is known as a \texttt{tag}.
  \item You have to manage the \emph{tag} field yourself.
\end{itemize}
}

% subsubsection pascal_variant_records_unions_ (end)
% subsection c_structure_declaration (end)