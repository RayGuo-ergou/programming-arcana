\clearpage
\subsection{Pascal Enumeration Declaration} % (fold)
\label{sub:pas_enum_declaration}

An enumeration is a list of available options for the type. A variable of an enumeration type can store one of these values. In Pascal you declare the enumeration by listing the available constants within parenthesis. 

\passyntax{psynt:enum-decl}{Enumeration Declarations}{type-decl/enum-decl}

\passection{\pascode{plst:test-enum}{Pascal code illustrating enumeration declarations.}{code/pascal/types/TestEnum.pas}}

\mynote{
\begin{itemize}
  \item This is the syntax that allows you to declare an enumeration in Pascal.
  \item The enumeration type contains a list of constants between parenthesis (i.e. \texttt{(\ldots)}). Each constant must have a unique name, and the by convention is all UPPERCASE.
\end{itemize}
}

% subsection c_enumeration_declaration (end)