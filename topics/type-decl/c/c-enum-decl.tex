\clearpage
\subsection{C Enumeration Declaration} % (fold)
\label{sub:c_enum_declaration}

An enumeration is a list of available options for the type. A variable of an enumeration type can store one of these values. In C you declare the enumeration using a \texttt{typedef}, and list the available constants within the braces. 

\csyntax{csynt:enum-decl}{Enumeration Declarations}{type-decl/enum-decl}

\csection{\ccode{clst:test-enum}{C code illustrating enumeration declarations.}{code/c/types/test-enum.c}}

\mynote{
\begin{itemize}
  \item This is the syntax that allows you to declare an enumeration in C.
  \item The declaration starts with \texttt{typedef}, which indicates that this is a declaration for a custom type, and \texttt{enum} indicating that this custom type is an enumeration.
  \item Following this is a list of constants between braces (i.e. \texttt{\{\ldots\}}). Each constant must have a unique name, and the by convention is all UPPERCASE.
  \item Finally the typedef ends with the name of the type you are declaring.
\end{itemize}
}

% subsection c_enumeration_declaration (end)