\subsection{Implementing Small DB in C} % (fold)
\label{sub:implementing_small_db_in_c}

\sref{sec:using_custom_types} of this Chapter introduced the Small DB program. A partial implementation of this program is shown in \lref{lst:c-small-db}. The type definitions, and code are missing the details needed to store and display double values. This program reads a number of rows of data from the user (determined by the \texttt{DB\_SIZE} constant). Each row stores a single value, being either a text value, an integer value, or a text value.

\straightcode{\ccode{lst:c-small-db}{C code for the Small DB program}{code/c/types/small-db.c}}

\mynote{
\begin{itemize}
  \item \texttt{strtol} is a function from the \texttt{stdlib.h} header. This converts the input to a integer, and updates \texttt{p} to point to the character the conversion got to before ending. Checking that this matches the end of string terminator (\csnipet{'\\0'}) indicates that the input matches the entire string.
  \item \texttt{strtod} is a similar function that converts text to a double. As with \texttt{strtol}, this updates \texttt{p} to refer to the character it got in the conversion. If this refers to the end of the text then the data does contain a double.
  \item The \texttt{trim} function removes spaces from the start and the end of the input, ensuring that numbers with leading or trailing spaces are still detected as numbers.
  \item The type definitions must appear before they are used, as a result they commonly appear at the start of the code after the includes.
  \item \texttt{limits.h} gives access to the constants \texttt{INT\_MAX} and \texttt{INT\_MIN}, these can be used to check if the integer read is in the range of an integer value.
  \item \texttt{errno.h} gives access to the \texttt{errno} global variable that contains an error number when the integer is outside of the range of a long.
  \item The code in \texttt{Read Row} includes code that skips any unwanted input on the current line. This requires two \texttt{scanf} calls: the first reading in any characters other than a newline, and the next reading in the newline. This method of skipping unwanted data in input is safe and works across multiple platforms.
  \item A number of online resources refer to using \texttt{fflush} to skip processing input between reads. This \textbf{should not} be used. The \texttt{fflush} function is used to ensure that output is written to the underlying hardware. The behaviour of this function \textbf{is not defined} for input streams. As a result it will not work as expected on all platforms.
\end{itemize}
}

% subsection implementing_small_db_in_c (end)