\clearpage
\subsection{Pascal Procedure Declaration} % (fold)
\label{sub:pas_procedure_declaration}

The syntax for a Pascal procedure declaration is shown in \fref{passynt:procedure-decl-procedure-decl}.

\passyntax{passynt:procedure-decl-procedure-decl}{a procedure}{procedure-decl/procedure-decl}

\passection{\pascode{lst:program-pas-print-steps}{Cooking a meal}{code/pascal/procedure-decl/PrintSteps.pas}}

\mynote{
\begin{itemize}
  \item There are four procedures declared in the code in \lref{lst:program-pas-print-steps}.
  \item A \textbf{procedure declaration} starts with the word \textbf{\texttt{procedure}}. This indicates to the compiler that the following code is a procedure declaration.
  \item The \textbf{procedure's name} is an identifier. This can be any valid \nameref{sub:pas_identifier} that has not been used before.
  \item The empty parenthesis must appear after the procedure's name, and before the semicolon and the \emph{block}.
  \bigskip
  \item There are a number of conventions, called coding standards, that describe how your code should appear for a given language. In this text we will use a common Pascal convention of having all \emph{procedure names} in \textbf{Pascal Case}, this uses an uppercase character for the first letter of each word in the identifier. So the \emph{Get Ingredients} procedure becomes \texttt{GetIngregients}.
\end{itemize}
}

% subsection c_procedure_declaration (end)