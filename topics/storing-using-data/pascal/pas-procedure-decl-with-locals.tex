\clearpage
\subsection{Pascal Procedure Declaration (with Local Variables)} % (fold)
\label{sub:pas_procedure_declaration_with_local_variables_}

The functions and procedures in Pascal can contain declaration for \nameref{sub:local_variable}s.

\passyntax{passynt:storing-using-data-procedure-decl}{Procedure Declaration (with Local Variables)}{storing-using-data/procedure-decl-with-locals}

\mynote{
\begin{itemize}
  \item This is the syntax for declaring \nameref{sub:local_variable}s in a procedure.
  \item See \lref{lst:variable-test-pas} for an example of declaring local variables.
  \item In \lref{lst:variable-test-pas} \ldots
  \begin{itemize}
    \item The \texttt{test} procedure has two local variables: \texttt{myLocal} and \texttt{anotherLocal}.
    \item The \texttt{main} function has one local variable called \texttt{localInt}.
  \end{itemize}
  \item The local variables are declared before the procedure's statements.
  \item In this text we will use a common Pascal convention of having all \emph{local variables} in \textbf{camel Case}, where the first character is lower case but subsequent words in the identifier start with an upper case character. So the \emph{My Name} local variable becomes \texttt{myName}.
  
\end{itemize}
}

% subsection c_procedure_declaration_with_local_variables_ (end)