\clearpage
\subsection{Pascal Terminal Input} % (fold)
\label{sub:pascal_terminal_input}

Pascal comes with a range of \nameref{sub:library}s that provide reusable programming artefacts, including reusable \nameref{sub:function} and \nameref{sub:procedure}s. The \texttt{System} unit includes procedures to read input from the Terminal. The \texttt{ReadLn} procedure is used to read data from the Terminal.

\begin{table}[h]
  \centering
  \begin{tabular}{|c|p{9cm}|}
    \hline
    \multicolumn{2}{|c|}{\textbf{Procedure Prototype}} \\
    \hline
    \multicolumn{2}{|c|}{} \\
    \multicolumn{2}{|c|}{\texttt{procedure ReadLn( {\ldots} )}} \\
    \multicolumn{2}{|c|}{} \\
    \hline
    \hline
    \textbf{Parameter} & \textbf{Description} \\
    \hline
    \texttt{ \ldots } & The \texttt{ReadLn} procedure takes a variable number of parameters. A value is read from the Terminal for each parameters. \\
    \hline
  \end{tabular}
  \caption{Parameters that must be passed to \texttt{ReadLn}}
  \label{tbl:readln parameters}
\end{table}

\passection{\pascode{lst:parameter-hello-name}{Example of reading data from the Terminal}{code/pascal/storing-using-data/HelloName.pas}}

% subsection c_terminal_input (end)