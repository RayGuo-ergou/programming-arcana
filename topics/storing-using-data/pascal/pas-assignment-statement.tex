\clearpage
\subsection{Pascal Assignment Statement} % (fold)
\label{sub:pas_assignment_statement}

The assignment statement is used to store a value in a variable.

\passyntax{passynt:storing-using-data-assignment-statement}{an assignment statement}{storing-using-data/assignment-statement}

\mynote{
\begin{itemize}
  \item This is the Pascal syntax for the \nameref{sub:assignment_statement}.
  \item In Pascal assignment is indicated by a colon followed by an equals sign ( \texttt{:=} ).
  \item The \emph{left hand side} of the assignment must be a valid variable, this is where the value is to be stored.
  \item The \emph{right hand side} of the assignment is an expression, this calculates the value that will be stored in the variable.
  \item There are multiple versions of the assignment, giving short hand ways of using the current value.
  \begin{itemize}
    \item \textbf{\texttt{=}} stores the value of the expression in the variable.
    \item \textbf{\texttt{+=}} increments the variable's value, \newline \passnipet{a += n;} is equivalent to \passnipet{a := a + n;}
    \item \textbf{\texttt{-=}} decrements the variable's value, \newline \passnipet{a -= n;} is equivalent to \passnipet{a := a - n;}
    \item \textbf{\texttt{*=}} multiplies the value in the variable by a factor. \newline \passnipet{a *= n;} is equivalent to \passnipet{a := a * n;}
    \item \textbf{\texttt{/=}} divides the value in the variable by a factor. \newline \passnipet{a /= n;} is equivalent to \passnipet{a := a / n;}
  \end{itemize}
\end{itemize}
}

\clearpage

\passection{\pascode{lst:assignment-test-pas}{Sample assignment statements}{code/pascal/storing-using-data/AssignmentTest.pas}}


% subsection c_assignment_statement (end)