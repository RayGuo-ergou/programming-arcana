\clearpage
\subsection{C Variable Declaration} % (fold)
\label{sub:c_variable_declaration}

A Variable Declaration allows you to create a Variable in your Code. In C you can declare variables in the Program's code, and in Functions and Procedures. 

\csyntax{csynt:storing-using-data-variable-decl}{Variable Declaration}{storing-using-data/variable-declaration}

\csection{\ccode{lst:variable-test-c}{Variable Declaration Tests}{code/c/storing-using-data/variable_test.c}}

\mynote{
\begin{itemize}
  \item This is the C Syntax for creating your own \nameref{sub:variable}.
  \item This syntax can be used to declare \ldots
  \begin{itemize}
    \item Local Variables within Functions and Procedures.
    \item Global Variables within the Program.
  \end{itemize} 
  \item In C the Variable Declaration starts with the \nameref{sub:type} name indicating the kind of data that will be stored.
  \item Following the Type is a list of the identifiers for the Variables that are being created. You can create one or more variables in a single Variable Declaration, but all of these Variables will have the same type.
  \item Each variable can be assigned a value when it is declared.
  \item The \textbf{const} modifier can be added to the start of a Variable declaration to create a Constant.
  \item See \nameref{sub:c_procedure_declaration_with_local_variables_} for details on declaring Local Variables within Functions and Procedures.
  \item See \nameref{sub:c_program_with_global_variables} for details on declaring Global Variables within the Program itself.
  \item The syntax for declaring Parameters is very similar, see \nameref{sub:c_procedure_declaration_with_parameters_}.
\end{itemize}
}

% subsection c_variable_declaration (end)