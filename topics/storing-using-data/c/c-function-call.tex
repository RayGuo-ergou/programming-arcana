\clearpage
\subsection{C Function Call} % (fold)
\label{sub:c_function_call}

\csyntax{csynt:function-decl-function-call}{a Function Call}{function-decl/function-call}

\csection{\ccode{clst:test-fn-call}{Example of Function Calls.}{code/c/storing-using-data/test-fn-calls.c}}

\mynote{
\begin{itemize}
  \item A C function call is similar to a \nameref{sub:procedure call}.
  \item You use the name of the \nameref{sub:function}, its identifier, to indicate which procedure is called.
  \item Following the Function's name is the list of \emph{arguments}, these are the values (or variables) that are being passed to the called Function.
  \item The return type of the Function determines where the Function may be called. 
  \begin{itemize}
    \item Procedure, \texttt{void} Functions, can only be called in a \nameref{sub:procedure call} \nameref{sub:statement}.
    \item Other Functions can be used in Expressions, see \nameref{sub:expressions_with_variables_}. In these cases the type of data returned by the Function will determine the type of the Function Call.
  \end{itemize}
  \item In Listing \ref{clst:test-fn-call} the values of the inner function calls are passed to the arguments of the outer calls. This means that \texttt{square(5)} is calculated first then \texttt{sqaure(4)}. The results of these two Function Calls are then passed as the \emph{arguments} into the call to the \texttt{sum} Function. In this case the Function call will be \texttt{sum(25, 16)} with \texttt{25} being the result returned by \texttt{square(5)} and \texttt{16} being the result returned by \texttt{square(4)}.
\end{itemize}
}
% subsection c_procedure_declaration (end)

