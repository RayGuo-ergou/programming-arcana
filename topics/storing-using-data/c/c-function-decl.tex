\clearpage
\subsection{C Function Declaration} % (fold)
\label{sub:c_function_declaration}

\csyntax{csynt:function-decl-function-decl}{a Function}{function-decl/function-decl}

\csection{\ccode{clst:square}{Example Function Declaration of a \texttt{square} Function.}{code/c/storing-using-data/test-square.c}}

\mynote{
\begin{itemize}
  \item In C \nameref{sub:function} and \nameref{sub:procedure} declarations are very similar. 
  \item In C, a Function's declaration starts with the \nameref{sub:type} of data the Function will return.
  \item This if followed by the name of the Function, and its Parameters. In the same way as is done in the \nameref{sub:c_procedure_declaration_with_parameters_}.
  \item The body of the program is a \texttt{block}, in the same was as a \nameref{sub:c_procedure_declaration_with_local_variables_}.
  \item See \nameref{sub:c_function_call} for the Syntax needed to call your Functions.
  \item See the \nameref{sub:return_statement} to see how to return a result from a Function in C.
  \item \texttt{void} is a type, so in C Functions and Procedures are identical. See \nameref{sub:c_procedure_declaration_as_function_} to see how C handles Procedures.
  \item The entry point of the Program is the \texttt{main} Function. It returns a number to the Operating System that can be used to indicate the success or failure of the program. You can read the value returned from the last program to execute in the Terminal using \bashsnipet{echo \$?}.
\end{itemize}
}

% subsection c_procedure_declaration (end)