\clearpage
\subsection{C Program (with Global Variables and Constants)} % (fold)
\label{sub:c_program_with_global_variables}

You can declare global variables and constants within a C Program file.

\csyntax{csynt:storing-using-data-program}{a program (with global variables and constants)}{storing-using-data/program-with-globals}

\mynote{
\begin{itemize}
  \item This syntax allows you to declare \nameref{sub:global_variable}s and Constants.
  \item See Listing \ref{lst:variable-test-c} for an example of declaring Global Variable and Constants.
  \item In Listing \ref{lst:variable-test-c} \ldots
  \begin{itemize}
    \item \texttt{global\_float} and \texttt{global\_int} are Global Variables. These can be accessed in both the \texttt{test} procedure and \texttt{main}.
    \item \texttt{PI} is a Global Constant, with the value 3.1415. This can be read in both the \texttt{test} procedure and \texttt{main}.
  \end{itemize}
  \item Global variable should be avoided.
  \item Global constants can be declared in two ways, using a \nameref{sub:c_variable_declaration} with the \texttt{const} modifier or using \texttt{\#define}.
  \item See \lref{clst:def-consts} for an example of \texttt{\#define} and constants.
  \item With \texttt{\#define} you are defining a `value' for the identifier, this identifier is then substituted with the `value' throughout your code. This means you have to pay particular attention to what `value' you use. 
  \item There are a number of conventions, called coding standards, that describe how your code should appear for a given language. In this text we will use a common C convention of having all \emph{Constants} in \textbf{UPPER CASE}, with underscores ( \_ ) used to separate words. So the \emph{Maximum Height} constant becomes \texttt{MAXIMUM\_HEIGHT}.
\end{itemize}
}

\clearpage

\csection{\ccode{clst:def-consts}{C program with a defined constant, and constant variable}{code/c/storing-using-data/defined-consts.c}}

% subsection c_program_with_global_variables (end)