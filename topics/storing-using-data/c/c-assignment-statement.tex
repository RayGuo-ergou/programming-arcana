\clearpage
\subsection{C Assignment Statement} % (fold)
\label{sub:c_assignment_statement}

The assignment statement is used to store a value in a variable.

\csyntax{csynt:storing-using-data-assignment-statement}{an Assignment Statement}{storing-using-data/assignment-statement}

\mynote{
\begin{itemize}
  \item This is the C syntax for the \nameref{sub:assignment_statement}.
  \item In C assignment is indicated by the equals sign ( = ).
  \item The \emph{left hand side} of the assignment must be a valid variable, this is where the value is to be stored.
  \item The \emph{right hand side} of the assignment is an expression, this calculates the value that will be stored in the Variable.
  \item There are multiple versions of the assignment, giving short hand ways of using the current value.
  \begin{itemize}
    \item \textbf{\texttt{=}} is the standard assignment, this stores the value of the expression in the Variable.
    \item \textbf{\texttt{+=}} increments the variable's value, \newline \csnipet{a += n;} is equivalent to \csnipet{a = a + n;}
    \item \textbf{\texttt{-=}} decrements the variable's value, \newline \csnipet{a -= n;} is equivalent to \csnipet{a = a - n;}
    \item \textbf{\texttt{*=}} multiplies the value in the variable by a given factor. \newline \csnipet{a *= n;} is equivalent to \csnipet{a = a * n;}
    \item \textbf{\texttt{/=}} divides the value in the variable by a factor. \newline \csnipet{a /= n;} is equivalent to \csnipet{a = a / n;}
  \end{itemize}
  \item The \texttt{++} and \texttt{-{-}} operators allow a variables value to be increments or decremented.
\end{itemize}
}

\csection{\ccode{lst:assignment-test-c}{Assignment Tests}{code/c/storing-using-data/assignment-test.c}}


% subsection c_assignment_statement (end)