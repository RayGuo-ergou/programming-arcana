\clearpage
\subsection{C Type Declarations (with pointers)} % (fold)
\label{sub:c_type_decl_with_ptrs}

In C you can declare custom types that make use of pointers. This includes alias types (see \nameref{sub:c_type_declaration}), structs (see \nameref{sub:c_structure_declaration}), and unions (see \nameref{sub:c_union_declaration}).

\csyntax{csynt:type-decl-with-ptr}{array and alias type declarations with pointers}{dynamic-memory/type-decl-with-ptrs}

\mynote{
\begin{itemize}
  \item This is the C syntax to for custom alias type that include \nameref{sub:pointer}s.
  \item The main difference is the inclusion of the \texttt{*} in the \emph{direct type declaration}. This indicates that the custom type can alias pointer types. This would allow you to declare a type such as \texttt{person\_ptr} that is a pointer to a person.
  \item  The inner type declaration allows you to have array of pointers. In these cases you use the brackets to indicate if you want to declare a pointer to an array, or an array of pointers. 
\end{itemize}
}

\begin{figure}[p]
\csection{\ccode{clst:test-type-alias-ptr}{C code demonstrating type aliasing with pointers}{code/c/dynamic-memory/alias-types-with-ptrs.c}}  
\end{figure}


% subsection c_array_declaration (end)