\clearpage

\subsection{C File Type} % (fold)
\label{sub:c_file_type}

C includes a \texttt{FILE} type that is used to interact with files on the computer. This type includes all of the information that C needs to read and write data to a file. 

\csection{\ccode{clst:file}{Example use of the FILE type to read and write text data}{code/c/file-io/text-file-io.c}}

\mynote{
\begin{itemize}
  \item The \texttt{FILE} type is used to read and write data from a file.
  \item You open the file using \texttt{fopen}.
  \item After the file has been opened you must remember to close it using \texttt{fclose}.
  \item File based versions of \texttt{printf} and \texttt{scanf} allow you to read and write data from the file.
  \item The IO code in C always works with \texttt{FILE} pointers, as a result all of your \texttt{FILE} variables will be pointers.
\end{itemize}
}


% subsection c_file_type (end)