\clearpage
\subsection{Pascal File Procedures} % (fold)
\label{sub:pas_file_functions}

There are a number of functions and procedures in Pascal that will give you the ability to read and write data from files.

\subsubsection{Assigning a filename} % (fold)
\label{ssub:assigning_a_filename}

Before you can interact with a file the first step will be to assign a filename to the \texttt{Text} variable.

\begin{table}[h]
  \centering
  \begin{tabular}{|c|p{9.5cm}|}
    \hline
    \multicolumn{2}{|c|}{\textbf{Procedure Prototype}} \\
    \hline
    \multicolumn{2}{|c|}{} \\
    \multicolumn{2}{|c|}{\texttt{procedure Assign( var fileVar: Text; filename: String )}} \\
    \multicolumn{2}{|c|}{} \\
    \hline
    \textbf{Parameter} & \textbf{Description} \\
    \hline
    \texttt{ fileVar } & The file variable to have its filename set\\
    & \\
    \texttt{ filename } & The name of the file to open. This can include a relative or absolute path to the file. \\
    \hline
  \end{tabular}
  \caption{Details of the \texttt{Assign} procedure}
  \label{tbl:assign}
\end{table}

% subsubsection assigning_a_filename (end)

\subsubsection{Opening the file to read} % (fold)
\label{ssub:opening_the_file_to_read}

The \texttt{Reset} procedure is used to reset the file cursor to the start of the file for reading.

\begin{table}[h]
  \centering
  \begin{tabular}{|c|p{9.5cm}|}
    \hline
    \multicolumn{2}{|c|}{\textbf{Procedure Prototype}} \\
    \hline
    \multicolumn{2}{|c|}{} \\
    \multicolumn{2}{|c|}{\texttt{procedure Reset( var fileVar: Text )}} \\
    \multicolumn{2}{|c|}{} \\
    \hline
    \textbf{Parameter} & \textbf{Description} \\
    \hline
    \texttt{ fileVar } & The file to reset, after this call you can read from this file.\\
    \hline
  \end{tabular}
  \caption{Details of the \texttt{Reset} procedure}
  \label{tbl:reset}
\end{table}

% subsubsection opening_the_file_to_read (end)

\subsubsection{Opening the file to write} % (fold)
\label{ssub:opening_the_file_to_write}

You can call either \texttt{Rewrite} or \texttt{Append} to write to the file. Rewrite deletes the old file contents, append moves to the end of the file and adds new data there.

\begin{table}[h]
  \centering
  \begin{tabular}{|c|p{9.5cm}|}
    \hline
    \multicolumn{2}{|c|}{\textbf{Procedure Prototype}} \\
    \hline
    \multicolumn{2}{|c|}{} \\
    \multicolumn{2}{|c|}{\texttt{procedure Rewrite( var fileVar: Text )}} \\
    \multicolumn{2}{|c|}{} \\
    \hline
    \textbf{Parameter} & \textbf{Description} \\
    \hline
    \texttt{ fileVar } & The file to rewrite. After this call you can write data to this file, this will override the existing contents.\\
    \hline
  \end{tabular}
  \caption{Details of the \texttt{Rewrite} procedure}
  \label{tbl:reset}
\end{table}

\clearpage

\texttt{Append} can also be used to open a file for write access. This will not overwrite existing data, keeping the cursor at the end of the existing file.

\begin{table}[h]
  \centering
  \begin{tabular}{|c|p{9.5cm}|}
    \hline
    \multicolumn{2}{|c|}{\textbf{Procedure Prototype}} \\
    \hline
    \multicolumn{2}{|c|}{} \\
    \multicolumn{2}{|c|}{\texttt{procedure Append( var fileVar: Text )}} \\
    \multicolumn{2}{|c|}{} \\
    \hline
    \textbf{Parameter} & \textbf{Description} \\
    \hline
    \texttt{ fileVar } & The file to append data to. After this call you can write data to this file and it will appear after the existing contents of the file.\\
    \hline
  \end{tabular}
  \caption{Details of the \texttt{Rewrite} procedure}
  \label{tbl:reset}
\end{table}

% subsubsection opening_the_file_to_write (end)

\subsubsection{Closing a file} % (fold)
\label{ssub:closing_a_file}

Once you have opened a file it is important that you also close it. The \texttt{Close} procedure can be used to close an opened file.

\begin{table}[h]
  \centering
  \begin{tabular}{|c|p{9.5cm}|}
    \hline
    \multicolumn{2}{|c|}{\textbf{Procedure Prototype}} \\
    \hline
    \multicolumn{2}{|c|}{} \\
    \multicolumn{2}{|c|}{\texttt{procedure Close( var fileVar: Text )}} \\
    \multicolumn{2}{|c|}{} \\
    \hline
    \textbf{Parameter} & \textbf{Description} \\
    \hline
    \texttt{ fileVar } & The file to close. \\
    \hline
  \end{tabular}
  \caption{Details of the \texttt{Close} procedure}
  \label{tbl:close}
\end{table}

\mynote{
\begin{itemize}
  \item Make sure to close all opened files.
  \item Remember to check all paths through your code.
\end{itemize}
}

% subsubsection closing_a_file (end)
\clearpage
\subsubsection{Writing text data to file} % (fold)
\label{ssub:writing_text_data_to_file}

You can use \texttt{WriteLn} to write data to a \texttt{Text} file that has been opened with write capabilities.

\begin{table}[h]
  \centering
  \begin{tabular}{|c|p{9cm}|}
    \hline
    \multicolumn{2}{|c|}{\textbf{Procedure Prototype}} \\
    \hline
    \multicolumn{2}{|c|}{} \\
    \multicolumn{2}{|c|}{\texttt{procedure WriteLn(destination: Text; \ldots )}} \\
    \multicolumn{2}{|c|}{} \\
    \hline
    \textbf{Parameter} & \textbf{Description} \\
    \hline
    \texttt{ destination } & The Text file to write the output into.\\
    & \\
    \texttt{\ldots}   & The data to be written \\
    \hline
  \end{tabular}
  \caption{Parameters that must be passed to \texttt{WriteLn}}
  \label{tbl:FileWriteLn}
\end{table}

\mynote{
\begin{itemize}
  \item This will write the data as text to the file.
  \item The file must be opened with write permissions using \texttt{Rewrite} or \texttt{Append}
\end{itemize}
}

% subsubsection writing_data_to_file (end)

\subsubsection{Reading text data from file} % (fold)
\label{ssub:reading_text_data_from_file}

Reading text data from a file is similar to reading data from the Terminal or from a string.

\begin{table}[h]
  \centering
  \begin{tabular}{|c|p{9.5cm}|}
    \hline
    \multicolumn{2}{|c|}{\textbf{Procedure Prototype}} \\
    \hline
    \multicolumn{2}{|c|}{} \\
    \multicolumn{2}{|c|}{\texttt{procedure ReadLn(source: Text; \ldots )}} \\
    \multicolumn{2}{|c|}{} \\
    \hline
    \multicolumn{2}{|c|}{\textbf{Returns}} \\
    \hline
    \texttt{int} & The number of values read by \texttt{fscanf}. \\
    \hline
    \textbf{Parameter} & \textbf{Description} \\
    \hline
    \texttt{ source } & The Text file from which the input is read.\\
    & \\
    \texttt{\ldots}   & The variables into which the values will be read. \\
    \hline
  \end{tabular}
  \caption{Parameters that must be passed to \texttt{ReadLn}}
  \label{tbl:File Readln}
\end{table}

\mynote{
\begin{itemize}
  \item This will read text data from the file.
  \item The file must be opened with read permissions.
\end{itemize}
}

\passection{\pascode{plst:text_io}{Example code that demonstrates writing a value and reading it back from a text file}{code/pascal/file-io/TextIO.pas}}


% subsubsection reading_data_from_file (end)


% \subsubsection{Writing binary data to file} % (fold)
% \label{ssub:writing_binary_data_to_file}
% 
% The \texttt{fwrite} function allows you to write binary data to a file. This requires you to pass a pointer to your data, as well as the size and number of elements you want written.
% 
% \begin{table}[h]
%   \centering
%   \begin{tabular}{|c|p{9cm}|}
%     \hline
%     \multicolumn{2}{|c|}{\textbf{Procedure Prototype}} \\
%     \hline
%     \multicolumn{2}{|c|}{} \\
%     \multicolumn{2}{|c|}{\texttt{size\_t fwrite( const void *ptr, size\_t size, size\_t count, FILE *destination)}} \\
%     \multicolumn{2}{|c|}{} \\
%     \hline
%     \multicolumn{2}{|c|}{\textbf{Returns}} \\
%     \hline
%     \texttt{int} & The number of elements written to the \texttt{destination} by \texttt{fwrite}. If this does not equal the \texttt{count} parameter it indicates an error occurred writing the data to the file.\\
%     \hline
%     \textbf{Parameter} & \textbf{Description} \\
%     \hline
%     \texttt{ ptr } & A pointer to the data to be saved to the file.\\
%     & \\
%     \texttt{ size } & The size of each element to be saved.\\
%     & \\
%     \texttt{ count } & The number of elements to be saved to the file.\\
%     & \\
%     \texttt{ destination } & The FILE to write the output into.\\
%     & \\
%     \hline
%   \end{tabular}
%   \caption{Parameters that must be passed to \texttt{fwrite}}
%   \label{tbl:fwrite}
% \end{table}
% 
% \mynote{
% \begin{itemize}
%   \item The file must be opened with write permissions.
%   \item This will write the binary data to file from the values pointed to by the \texttt{prt} parameter.
% \end{itemize}
% }
% 
% \csection{\ccode{clst:write_binary}{Example code that writing an array of double values to a binary file.}{code/c/file-io/write_binary.c}}
% 
% 
% % subsubsection writing_binary_data_to_file (end)
% 
% \clearpage
% \subsubsection{Reading binary data from file} % (fold)
% \label{ssub:reading_binary_data_from_file}
% 
% To read back binary data you need to use \texttt{fread}. This reads back a block of data from the file, and stores it in memory at a location indicated by a pointer.
% 
% \begin{table}[h]
%   \centering
%   \begin{tabular}{|c|p{9cm}|}
%     \hline
%     \multicolumn{2}{|c|}{\textbf{Procedure Prototype}} \\
%     \hline
%     \multicolumn{2}{|c|}{} \\
%     \multicolumn{2}{|c|}{\texttt{size\_t fread( void *ptr, size\_t size, size\_t count, FILE *destination)}} \\
%     \multicolumn{2}{|c|}{} \\
%     \hline
%     \multicolumn{2}{|c|}{\textbf{Returns}} \\
%     \hline
%     \texttt{int} & The number of elements read from the \texttt{destination} by \texttt{WriteLn}. If this does not equal the \texttt{count} parameter it indicates an error occurred reading the data from the file.\\
%     \hline
%     \textbf{Parameter} & \textbf{Description} \\
%     \hline
%     \texttt{ ptr } & A pointer to the location to store the loaded data. This must be large enough to store the values loaded.\\
%     & \\
%     \texttt{ size } & The size of each element to be loaded.\\
%     & \\
%     \texttt{ count } & The number of elements to be loaded from the file.\\
%     & \\
%     \texttt{ destination } & The FILE to read the data from.\\
%     & \\
%     \hline
%   \end{tabular}
%   \caption{Parameters that must be passed to \texttt{fwrite}}
%   \label{tbl:fread}
% \end{table}
% 
% \mynote{
% \begin{itemize}
%   \item The file must be opened with read permissions.
%   \item You must ensure that \texttt{ptr} points to sufficient space to load the data into.
% \end{itemize}
% }
% 
% \begin{figure}[p]
%   \csection{\ccode{clst:read_binary}{Example code that reads an array of double values from a binary file.}{code/c/file-io/read_binary.c}}  
% \end{figure}
% 
% 
% % subsubsection reading_binary_data_from_file (end)
% 
% subsection c_file_functions (end)