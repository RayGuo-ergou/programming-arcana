\clearpage

\subsection{Pascal Text Type} % (fold)
\label{sub:pas_file_type}

Pascal includes \texttt{File} and \texttt{Text} types that are used to interact with files on the computer. The \texttt{File} type is used for binary files, the \texttt{Text} type is used for text files.

\passection{\pascode{plst:file}{Example use of the Text type to read and write text data}{code/pascal/file-io/TextFileIO.pas}}

\mynote{
\begin{itemize}
  \item The \texttt{Text} type is used to read and write text data from a file.
  \item You open the file for reading using \texttt{Assign} then \texttt{Reset}, to write use \texttt{Assign} then \texttt{Rewrite}.
  \item After the file has been opened you must remember to close it using \texttt{Close}.
  \item File based versions of \texttt{WriteLn} and \texttt{ReadLn} allow you to read and write text data from the file.
\end{itemize}
}


% subsection c_file_type (end)